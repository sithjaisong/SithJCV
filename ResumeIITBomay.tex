\documentclass[a4paper,10pt]{article}
%-----------------------------------------------------------
\usepackage[top=0.75in, bottom=0.75in, left=0.55in, right=0.85in]{geometry}
\usepackage{graphicx}
\usepackage{url}
\usepackage{palatino}
\usepackage{tabularx}
\fontfamily{SansSerif}
\selectfont

\usepackage[T1]{fontenc}
\usepackage
%[ansinew]
[utf8]
{inputenc}

\usepackage{color}
\definecolor{mygrey}{gray}{0.75}
\textheight=9.75in
\raggedbottom

\setlength{\tabcolsep}{0in}
\newcommand{\isep}{-2 pt}
\newcommand{\lsep}{-0.5cm}
\newcommand{\psep}{-0.6cm}
\renewcommand{\labelitemii}{$\circ$}

\pagestyle{empty}
%-----------------------------------------------------------
%Custom commands
\newcommand{\resitem}[1]{\item #1 \vspace{-2pt}}
\newcommand{\resheading}[1]{{\small \colorbox{mygrey}{\begin{minipage}{0.975\textwidth}{\textbf{#1 \vphantom{p\^{E}}}}\end{minipage}}}}
\newcommand{\ressubheading}[3]{
\begin{tabular*}{6.62in}{l @{\extracolsep{\fill}} r}
	\textsc{{\textbf{#1}}} & \textsc{\textit{[#2]}} \\
\end{tabular*}\vspace{-8pt}}
%-----------------------------------------------------------

\begin{document}
\hspace{0.5cm}\\[-0.2cm]

\textbf{Deepak Sahu} \\
\indent Nagar Palika Colony \\
\indent Piproda road, Nanakhedhi \\
\indent Guna (M.P., INDIA) - 473001 \\
\indent Email-id : \textbf{deepaksahu1808@gmail.com} \\
\indent Mobile No.: \textbf{09930154985} \\
\indent Alt Mob No.: \textbf{08655475283} \\

\resheading{\textbf{ACADEMIC DETAILS} }\\[\lsep]
\\ \\
%\begin{table}[ht!]
%\begin{center}
\indent \begin{tabular}{ l @{\hskip 0.15in} l @{\hskip 0.15in} l @{\hskip 0.15in} l @{\hskip 0.15in} l }
\hline
\textbf{Examination} & \textbf{University} & \textbf{Institute} & \textbf{Year} & \textbf{CPI/\%} \\
\hline
Post Graduate Specialization:\,\, & \textit{Computer Science and Engineering} \\
Post Graduation & IIT Bombay & IIT Bombay & 2014 & 6.38 \\
UnderGraduate Specialization: & \textit{Computer Engineering} \\
Graduation & DAVV, Indore & IET-DAVV, Indore & 2012 & 71.36 \\
Intermediate/+2 & CBSE & JNV Ashoknagar & 2007 & 77.80 \\
Matriculation & CBSE & JNV Ashoknagar & 2005 & 80.40 \\
\hline
\end{tabular}
%\end{center}
%\end{table}
\\ \\

\resheading{\textbf{FIELDS OF INTEREST} }\\[\lsep]
\begin{itemize}
\item \noindent Wireless Network and Network Security, Implementation Techniques in Relational Databases, Algorithms, Data Structures.
\end{itemize}

\resheading{\textbf{TECHNICAL SKILLS} }\\[\lsep]
\begin{itemize}
\item \noindent \textbf{Languages} (C, C++, Java),\textbf{Database} (MySQL) \textbf{Script} (Python, Shell), \textbf{Tools} (Eclipse, \LaTeX, Gnuplot).
\end{itemize}

\resheading{\textbf{MAJOR PROJECTS AND SEMINAR} }\\[\lsep]
\begin{itemize}
\item \textbf{Media Access Control in Wireless Network
} (M. Tech. Project) \\
 \emph{(Guide:Prof. Kameswari Chebrolu
, May'13 - till date)} \\[-0.6cm]
	\begin{itemize}\itemsep \isep
	\item Objective :Performance analysis of HTTP web browsing traffic in TDMA Mesh Networks with different
MAC protocols.
.
	\item Performance analysis will help in comparing different MAC protocols based on different network sce-
narios.

	\item Studied various papers related to different MAC protocols and now working on improving simulations.

	\item The work is revolve around new MAC protocol \emph{LiT MAC} and its comparison with available MAC pro-
tocols.

	\end{itemize}

\item \textbf{Object Recognition.
} (M. Tech. Seminar) \\
 \emph{(Guide: Prof. G. Nagaraja
, Jan'13 - Apr'13)} \\[-0.6cm]
	\begin{itemize}\itemsep \isep
	\item Understand the basic technique involve in object recognition.
	\item Understand the technical problems arises while recognizing mobile objects.
	\end{itemize}

\item \textbf{Data Structure Visualization.} (B.E. Project) \\ 
 \emph{(Guide: Mr. Jagdish Singh Raikwal)} 
 \\[-0.6cm]
	\begin{itemize}\itemsep \isep
	\item Objective : To depicts various data structures visually.
	\item Create visual display of binary search tree, stack and AVL tree.
	\end{itemize}
\end{itemize}

\resheading{\textbf{COURSE PROJECTS} (Group Projects)}\\[\lsep]

\begin{itemize}
\item \textbf{Implementation of optimized OR-IS-NULL join in postgres} (RDBMS)\\
 \emph{(Guide: Prof. S. Sudarshan
, July'12 - Nov'12)} \\[-0.6cm]
	\begin{itemize}\itemsep \isep
	\item Added special join type "\emph{merge-null join}" to redirect such queries to merge-join node in postgres.
	\item Extended merge join module to perform merge-null join within fewer number of passes.
	\item Further extended the code to handle multiple attributes with OR-IS-NULL clause.
	\item Impact: Reduced scans over input reduces the cost of join by large factor.
	\end{itemize}

\item \textbf{Dynamic Control Dependence.} (Advanced Compiler)\\
 \emph{(Guide: Prof. D. M. Dhamdhere
, July'12 - Nov'12)} \\[-0.6cm]
	\begin{itemize}\itemsep \isep
	\item Objective : Study and analyze previous work in the field of Dynamic Control dependence. The soul pur-
pose of the project is to come up with the idea of how the various statements depends on its predecessor
control statement.
	\end{itemize}

\item \textbf{Block cipher design and analysis.} (An Introduction to number theory and cryptography)
	\begin{itemize}\itemsep \isep
	\item We generate dynamic S-Box \& perform linear mixing of input bits

	\item Fiestel Structure is used for the operations.
	\end{itemize}

\item \textbf{Optimization Techniques in ECC} (Network Security and Cryptography)\\
 \emph{(Guide: Prof. Bernard L. Menezes
, Jan'13 - April'13)} \\[-0.6cm]
	\begin{itemize}\itemsep \isep
	\item Studied and compare different techniques for optimizing cost of scalar multiplication in ECC.
	\item Learn Non-Adjacent Form (NAF), Window-NAF (wNAF), Near Factorization (NF) and Double base \&
Multibase Number Representation.
	\end{itemize}

\end{itemize}

\resheading{\textbf{POSITIONS OF RESPONSIBILITY} }\\[\lsep]
\begin{itemize}\itemsep \isep
\item Work as a system administrator for three semester starting from July' 2012 in IIT-Bombay.

\item Work as Teaching Assistance of Prof. D.B. Phatak in HS699.
\end{itemize}

\resheading{\textbf{COURSES TAKEN} }\\[\lsep]
\begin{itemize}
\item Advanced Compiler, Implementation Techniques in Relational Database Management System, Program Analysis, Advanced Computer Architecture, Philosophy of Education. 
\item Introduction to Number Theory and Cryptography, Network Security and Cryptography, Communication Networks, Artificial Intelligence, Development of Mathematics in India.
\end{itemize}

\resheading{\textbf{ACHIEVEMENT} }\\[\lsep]
\begin{itemize}
\item \noindent In GATE 2012, secured All India Rank 261 with 99.83 percentile.
\end{itemize}

\resheading{\textbf{STRENGTHS} }\\[\lsep]
\begin{itemize}
\item \noindent Positive Attitude, Consistency, Cooperative with co-members.
\end{itemize}

\resheading{\textbf{INTEREST AND HOBBIES} }\\[\lsep]
\begin{itemize}
\item \noindent Solving Puzzles and Number Problems.
\item \noindent Playing Chess, Cricket, Table Tennis.

\end{itemize}

\end{document}

